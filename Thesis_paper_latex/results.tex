\chapter{Results}

% Characterizing flux differences 
% (Key figures: e.g. scatter plots, simulated flux maps, cofactor pie plots)


% Results Section 1. a figure panel with 
% - A: brief metabolic map; 
% - B: exchange fluxes;
% - C: main intracellular fluxes. Different GRs on x axis


\textit{Rhodotorula toruloides} can naturally accumulate high amounts of lipids, but the 
metabolic principles that make this possible and differentiate \textit{R. toruloides} from non-oleaginous yeast are not fully understood.
Biosynthesis of the main lipid precursors acetyl-CoA and NADPH takes place in the central carbon metabolism. A better understanding of which 
metabolic pathways are used in production of these precursors and thus contribute to lipid accumulation, would aid in designing better 
metabolic engineering strategies for increasing lipid production.

Genome-scale metabolic models contain all known biochemical reactions of the specific organism and allow the calculation of
metabolic fluxes, which represent the activity of metabolic pathways under specified conditions. 
This makes GEMs important tool for studying metabolism, but it is important that the predictive power of the GEM is
adequate. For \textit{R. toruloides} several genome-scale metabolic models are available, but so far comprehensive overview of 
simulations focused on central carbon metabolism with these models has not been presented.
This work will help in the future to design \textit{R. toruloides} as the leading microbial cell factory for production of microbial oils.


% \section{Intracellular fluxes and supply of NADPH} %


% Describe the fluxes in central carbon metabolism of R. toruloides;
%% Explain the preference towards XPK or ACL pathway for producing acetyl-CoA

% Extract cofactor balances and analyze what makes model to prefer one or the other.
% Explain where the NADPH is produced  

% For plotting, the enzymes that had fluxes lower than 2% were added to the 'Others' section


% Also cofactor balances of NADH and ATP were plotted.
% * For the model rhto-GEM, there is no difference between any cofactor balances on different growth rates nor objective function
% * iRhtoC - same as with rhto-GEM
% * IFO0880 - differences between the objective functions on the lowest growth rate: biomass max predicts alcohol dehydrogenase as the main source of NADPH (a loop??), NGAM min predicts Homoserine dehydrogenase
%  on the highest GR both obj functions predict alcohol dehydrogenase as the main source, consuming fluxes differ slightly on both GRs - but for us it's not important?
% NADH and ATP fluxes don't differ
% * IFO0880_jsb - NADPH: on lowest GR both obj functions predict fairly similarly that aldehyde dehydrogenase is the main source of NADPH
% on highest GR biomass max as obj predicts alcohol dehydrogenase and NGAM min predicts homoserine dehydrogenase
% NADH fluxes don't differ on different GR rates but they differ between biomass max and NGAM min as obj:
%  biomass max - predicts several enzymes, NGAM min redicts glyceraldehyde-phosphate dehydrogenase as the main source of NADH
% ATP don't differ between GR but between objective functions: biomass max predicts ATP synthase as the main source and NGAM min ADP/ATP transporter

% Differences between models:
% 

% \textbf{Biomass maximisation as objective function}

% % Solution fluxes of rhto-GEM showed that NADPH is recycled via ... . 
% Production and consumption of NADPH with the different models is shown in figures below.


% Four genome-scale metabolic models of \textit{Rhodotorula toruloides} - rhto-GEM, iRhto1108, Rt\_IFO0880 and Rt\_IFO0880\_LEBp2023 - 
% were used for simulating this yeast's metabolic flux distribution using flux balance analysis. 
% Different objective functions, maximisation of biomass and minimisation
% of NGAM, were tested under carbon limitation. 


% Fluxes through notable phosphoketolase and ATP-citrate lyase pathways, 
% that are known to be important for production of lipid precursors in olegeanous yeast, were in the center of interest and are distinguished in intracellular 
% flux figures from other enzymes using dashed line.
% What is more, lipid synthesis demands high amounts of NADPH, but only a few enzymes can generate it. 
% Pentose phosphate pathway and malic enzyme have been proposed as the main candidate enzymes for recycling NADPH in \textit{R. toruloides} 
% \cite{Ratledge2014}. Main intracellular fluxes of central carbon metabolism and production and consumption of NADPH were compared between the simulations 
% of four models. 


\section{Biomass maximisation as an objective function}

To obtain the flux distribution through the metabolic network, simulations using firstly biomass maximization as an objective function were carried out.
Calculations were made under carbon limitation constraining glucose 
uptake. To assess the flux patterns over different biomass growth rates, the 
solutions were constrained over five experimental glucose uptake rate values - $0.476, 1.114, 1.648, 2.305$ and $3.1$ $(mmol/gDW/h)$.

All simulated fluxes are available on a Github repository (\url{https://github.com/maivehanni/BSc_thesis}). 
In further analysis we focused on selected exchange and intracellular fluxes, which are visualized in graphs and shown below.

All figures show the fluxes of metabolites over growth rates from $0.05 - 0.25$ $1/h$.
Exchange fluxes plots show the exchange (uptake or secretion) rate of glucose, oxygen, ammonium, sulphate, phosphate and carbon dioxide.
The experimental exchange rates of glucose, oxygen and carbon dioxide are also visualized on the same plot and visualized with dashed lines.
Intracellular fluxes graphs show the fluxes of NGAM, glucose uptake, glucose 6-phosphate dehydrogenase (oxPPP), transketolase 1, transaldolase, 
transketolase 2, 
fructose-biphosphate aldolase, pyruvate decarboxylase, pyruvate dehydrogenase, phosphoketolase and ATP-citrate lyase.

NADPH producing and consuming fluxes were also investigated on growth rates from $0.05 - 0.25$ $1/h$, but only the ones that significantly differed on 
different growth rates, are shown below. In these pie charts, the upper part of the pie shows producing fluxes and the lower part shows consuming fluxes.
The flux of the enzyme is shown between the brackets after the name of the metabolite, and infront of the name is the percent
that the given enzyme makes up of the total producing or consuming NADPH flux.


\textbf{rhto-GEM}

Compared to the experimental physiological rate data, rhto-GEM predicts higher exchange fluxes for a measured growth rate (figure \ref{rhto-GEM biomass max}).
This result was expected as GEMs are known to overestimate fluxes per growth rate.

% Here, you should briefly describe other intracellular fluxes plotted as well. This goes like, “at the x5p branching point, the flux through TKT
% and TAL was predicted to be increased linearly from 0.5 to 3.0 mmol/gDCW/h, over the growth rates from 0.05 to 0.3 1/h, respectively. Fluxes of PDC and PDH 
% were predicted to be from X to Y mmol/gDCW/h, representing Z % of the carbon from pyruvate branching point.”

% % from branching point means you take the flux OUT from pyruvate and divide against the flux IN to pyruvate (flux of enolase).
% model.metabolites.metabolite_ID.summary() gives the percentages

% For other sections, you might not have to go through this boring description, if it repeats. But for the first time, you must describe everything you plotted.

All the investigated intracellular fluxes were predicted by the model to increase linearly over the growth rates from $0.03$ to $0.25$ $1/h$.
Flux through oxPPP was predicted to increase from $0.18$ to $1.45$ $mmol/gDCW/h$, representing around $48\%$ of the carbon from 
D-glucose 6-phosphate branching point. From the D-xylulose 5-phosphate branching point, $32\%$ of the flux was 
predicted to go to transketolase 1 (fluxes from $0.04$ to $0.34$), $28\%$ to 
transketolase 2 ($0.04-0.3$) and $40\%$ to phosphoketolase (fluxes from $0.05-0.4$). 
The flux of transaldolase was predicted to be zero on all rates. Fructose biphosphate aldolase was predicted to increase from 
$0.28$ to $1.15$ $mmol/gDCW/h$, representing $82\%$ of the carbon from D-fructose 6-phosphate. 
Pyruvate decarboxylase represents $15\%$ and pyruvate dehydrogenase $59\%$ of the carbon from % This percent is a bit complicated bc there's mode mithocondrial pyruvate then transported there by pyruvate transport, 4% is produced by malic enzyme
cytososlic pyruvate branching point having fluxes from $0.06$ to $0.5$ and $0.46-2.25$ $mmol/gDCW/h$, respectively.
The model did not predict any flux for ACL, instead it predicts that acetyl-CoA is produced through phosphoketolase. % for producing acetyl-CoA from citrate transported from mithocondria (TCA cycle), was zero. 

For NADPH production and consumption there are no differences 
between different growth rates. In all cases the model predicts that around $90\%$ of the NADPH is produced by glucose 6-phosphate dehydrogenase and
phosphogluconate dehydrogenase (oxPPP). Ca $6\%$ is produced by methylenetetrahydrofolate dehydrogenase (figure \ref{fig:rhtoGEM_bm_NADPH}).
\begin{figure}[H]
    \centering
    \includegraphics[width=\linewidth]{rhtoGEM_biomass_max.png}
    \caption{Exchange and intracellular fluxes in \textit{R. toruloides} with model rhto-GEM optimized for biomass maximization 
    and constrained over five glucose uptake rates.}
    \label{rhto-GEM biomass max}
\end{figure}
\begin{figure}[H]
    \centering
    \includegraphics[width=\linewidth]{rhtoGEM_bm_NADPH.png}
    \caption{NADPH producing and consuming fluxes in \textit{R. toruloides} with model rhto-GEM optimized for biomass maximization. Glucose uptake was constrained 
    to the lowest rate.}
    \label{fig:rhtoGEM_bm_NADPH}
\end{figure}



\textbf{iRhtoC}

This model predicts very similar exchange fluxes as the previous model, predicting higher rates than experimentally measured. 
But intracellular fluxes differ from model rhto-GEM. iRhtoC predicts that over the five growth rates
flux through oxPPP increase from $0.24$ to $1.69$ $mmol/gDCW/h$ (55\% from D-glucose 6-phosphate branching point), 
which is slightly more than rhto-GEM predicted.
Fluxes of transketolase 1 and 2 are also a bit higher than predicted by rhto-GEM ($0.08-0.56$ and $0.07-0.5$ representing 
53\% and 47\% from D-xylulose 5-phosphate branching point, respectively). 
Fluxes of transaldolase and pyruvate dehydrogenase are predicted to be zero. 
Fructose-bisphosphate aldolase (representing 93\% from D-fructose 6-phosphate) and pyruvate dehydrogenase (80\% from cytososlic pyruvate 
branching point) 
fluxes are from $0.24$ to $1.49$ and from $0.53$ to $3.29$ $mmol/gDCW/h$, respectively.
ATP-citrate lyase has a flux from $0.18$ to $1.29$ representing 40\% of carbon from mithocondrial citrate (figure \ref{iRhtoC_biomass_max}). 

This model predicts that almost $90\%$ of NADPH is produced by glucose 6-phosphate dehydrogenase (oxPPP) and
phosphogluconate dehydrogenase on all rates, but around $6\%$ is produced by by malic enzyme on growth rates $0.03$ and $0.23$ and on other 
rates the model predicts isocitrate dehydrogenase instead (figures \ref{fig:iRhtoC_bm_NADPH0} and \ref{fig:iRhtoC_bm_NADPH1}). 
%Predicted consuming fluxes differ from rhto-GEM predictions by the use of malic enzyme and isocitrate dehydrogenase.
\begin{figure}[H]
    \centering
    \includegraphics[width=\linewidth]{iRhtoC_biomass_max.png}
    \caption{Exchange and intracellular fluxes in \textit{R. toruloides} with model iRhtoC optimized for biomass maximization 
    and constrained over five glucose uptake rates.}
    \label{iRhtoC_biomass_max}
\end{figure}
\begin{figure}[H]
    \centering
    \includegraphics[width=\linewidth]{iRhtoC_bm_NADPH_min.png}
    \caption{NADPH producing and consuming fluxes in \textit{R. toruloides} with model iRhtoC. The model was optimized for biomass maximization and glucose 
    uptake was constrained to the lowest rate. The fluxes are same when glucose uptake is constrained to highest rate.}
    \label{fig:iRhtoC_bm_NADPH0}
\end{figure}
\begin{figure}[H]
    \centering
    \includegraphics[width=\linewidth]{iRhtoC_bm_NADPH1.png}
    \caption{NADPH producing and consuming fluxes in \textit{R. toruloides} with model iRhtoC. The model was optimized for biomass maximization. 
    Results are the same when glucose uptake is constrained to $1.114, 1.648$ or $2.305$ $(mmol/gDW/h)$.}
    \label{fig:iRhtoC_bm_NADPH1}
\end{figure}


\textbf{Rt\_IFO0880}

Rt\_IFO0880 also predicts higher exchange fluxes than experimentally measured. 
This model has two phosphoketolases, fructose-6-phosphate phosphoketolase (FPK) and xylulose-5-phosphate phosphoketolase (XPK), 
but as models rhto-GEM and iRhtoC
have one phosphoketolase, FPK and XPK fluxes have been summed together for easier comparison with other models.
Model predictions differ from models rhto-GEM and iRhtoC by not having any flux in oxPPP and transketolase 2.
Transketolase 1 represents 13\% of the carbon from glyceraldehyde 3-phosphate. Transaldolase represents 19\%, fructose-bisphosphate 
aldolase 52\% and FPK 21\% carbon from D-Glucose 6-phosphate. 
XPK represents 100\% of carbon from D-xylulose 5-phosphate. Summed flux of XPK and FPK is from $0.18$ to $1.25$ over the rates.
Pyruvate decarboxylase represents 7\% and pyruvate 
dehydrogenase 58\% of carbon from cytososlic pyruvate branching point (figure \ref{fig:IFO0880_biomass_max}).
Similarly to rhto-GEM, this model also predicts the use of phosphoketolase and no use of ACL. 

In NADPH production and consumption on different rates there are not significant differences, as
around $90\%$ of NADPH is produced by alcohol dehydrogenase on all rates (figure \ref{fig:IFO0880_biomass_max_NADPH_max}). 
This is very different from other models.
%(That indicates unrealistic loops)
\begin{figure}[H]
    \centering
    \includegraphics[width=\linewidth]{IFO0880_biomass_max.png}
    \caption{Exchange and intracellular fluxes in \textit{R. toruloides} with model Rt\_IFO0880 optimized for biomass maximization 
    and constrained over five glucose uptake rates.}
    \label{fig:IFO0880_biomass_max}
\end{figure}
\begin{figure}[H]
    \centering
    \includegraphics[width=\linewidth]{IFO0880_biomass_max_NADPH.png}
    \caption{NADPH producing and consuming fluxes in \textit{R. toruloides} with model Rt\_IFO0880. The model was optimized for biomass maximization 
    and glucose uptake was constrained.}
    \label{fig:IFO0880_biomass_max_NADPH_max}
\end{figure}



\textbf{Rt\_IFO0880\_LEBp2023}

Exchange fluxes align with other models. From D-glucose 6-phosphate branching point 9\% of the carbon is represented by oxPPP. 
Transketolase 1 and 2 represent very
low carbon percentage, only 0.3\% and 1.3\%, from glyceraldehyde 3-phosphate branching point, respectively. Transaldolase represents
0.6\%, transketolase 2 2\% and fructose-bisphosphate aldolase 89\% from D-glucose 6-phosphate. Pyruvate decarboxylase represents 5\% 
and pyruvate dehydrogenase 62\% carbon from cytososlic pyruvate. %cytosolic
33\% of carbon from mithocondrial citrate is represented by ACL. (\ref{fig:Rt_IFO0880_LEBp2023_biomass_max})
On all growth rates the use of ACL is predicted instead of phosphoketolase, which is similar with the model iRhtoC. 
(As this model is an updated version of Rt\_IFO0880, it also has two phosphoketolases - FPK and XPK, which fluxes
have been summed together and represented as phosphoketolase.) 

On lowest growth rate, NADPH is predicted to be produced mainly ($\sim 85\%$) by aldehyde dehydrogenase and on all other 
rates mainly ($\sim 75\%$) by alcohol 
dehydrogenase and $\sim 15\%$ by glucose-6-phosphate dehydrogenase and phosphogluconate dehydrogenase (figures \ref{fig:Rt_IFO0880_LEBp2023_biomass_max_NADPH0} 
and \ref{fig:Rt_IFO0880_LEBp2023_biomass_max_NADPH1}). 
Interestingly, as the growth rate increases the use of aldehyde dehydrogenase slightly decreases and use of oxidative pentose phosphate pathway for NADPH production increases.
% (That indicates unrealistic loops?)


\begin{figure}[H]
    \centering
    \includegraphics[width=\linewidth]{Rt_IFO0880_LEBp2023_biomass_max.png}
    \caption{Exchange and intracellular fluxes in \textit{R. toruloides} with model Rt\_IFO0880\_LEBp2023 optimized for biomass maximization 
    and constrained over five glucose uptake rates.}
    \label{fig:Rt_IFO0880_LEBp2023_biomass_max}
\end{figure}
\begin{figure}[H]
    \centering
    \includegraphics[width=\linewidth]{Rt_IFO0880_LEBp2023_biomass_max_NADPH.png}
    \caption{NADPH producing and consuming fluxes in \textit{R. toruloides} with model Rt\_IFO0880\_LEBp2023. The model was optimized for biomass maximization. 
    Glucose uptake was constrained on the lowest rate.}
    \label{fig:Rt_IFO0880_LEBp2023_biomass_max_NADPH0}
\end{figure}
\begin{figure}[H]
    \centering
    \includegraphics[width=\linewidth]{Rt_IFO0880_LEBp2023_biomass_max_NADPH1.png}
    \caption{NADPH producing and consuming fluxes in \textit{R. toruloides} with model Rt\_IFO0880\_LEBp2023. The model was optimized for biomass maximization 
    and glucose uptake constrained to $1.114$. 
    The results are very similar when glucose uptake is constrained to $1.114, 1.648, 2.305$ or $3.1$.}
    \label{fig:Rt_IFO0880_LEBp2023_biomass_max_NADPH1}
\end{figure}

Simulations with biomass maximisation as objective function under carbon limitation, showed that the models prefer different pathways for 
production of acetyl-CoA. The models rhto-GEM and Rt\_IFO0880 produced acetyl-CoA through phosphoketolase, whereas models iRhtoC and the modified model 
of Rt\_IFO0880 (Rt\_IFO0880\_LEBp2023) produced acetyl-CoA using ACL. NADPH sources also differ. On all growth rates,
 models rhto-GEM and iRhtoC predict that
most of the NADPH is produced by the oxidative pentose phosphate pathway, but $\sim 6\%$ by methylenetetrahydrofolate dehydrogenase 
in rhto-GEM and by malic enzyme in iRhtoC. Model Rt\_IFO0880 predicts that most of the NADPH is produced by alcohol dehydrogenase
and Rt\_IFO0880\_LEBp2023 predicts that on the lowest growth rate NADPH is mainly ($\sim 85\%$) produced through aldehyde dehydrogenase 
and on all other rates, $\sim 75\%$ is produced by alcohol dehydrogenase and $\sim 15\%$ by oxPPP.


\section{Non-growth associated maintenance minimisation as an objective function}

NGAM minimisation as an objective function was investigated to see whether the use of different objective function gives distinct results.
Specifically, whether Rt\_IFO0880-based models could re-arrange NADPH regeneration through different pathways than alcohol and aldehyde dehydrogenase. Optimizing for NGAM minimisation it is assumed that cell strive for satisfying physiological parameters with least energy expenditure as NGAM minimisation decreases the ATP demand.
% Can you add one sentence on what exactly this obj. function does? Optimizing for.. under assumption that cell 
% will strive for satisfying physiological parameters with least energy expenditure?

Same experimentally measured exchange and secretion rates were used throughout the simulations. This objective function also needed constraints 
on biomass growth rate because otherwise the simulation chooses zero as its flux. 
Growth rate was constrained to the values that each model
predicted in the simulations optimized for biomass maximisation, because experimental growth rates together with experimental glucose 
uptakes were infeasible for the models. 

\textbf{rhto-GEM}

Model rhto-GEM optimized for NGAM minimisation predicts similar fluxes as with 
previous objective function. 

\textbf{iRhtoC}

iRhtoC optimized for NGAM minimisation predicts same exchange and intracellular fluxes as with 
previous objective function. 
As previously, most of the NADPH is produced by oxPPP, but on the lowest growth rate $\sim 6\%$ is produced by isocitrate dehydrogenase and on all other 
rates by malic enzyme (figures \ref{fig:iRhtoC_nm_NADPH_min} and \ref{fig:iRhtoC_nm_NADPH1}). NGAM minimisation and biomass maximisation as an objective 
function predict differences in the use of malic enzyme vs isocitrate dehydrogenase on different growth rates.

\begin{figure}[H]
    \centering
    \includegraphics[width=\linewidth]{iRhtoC_nm_NADPH_min.png}
    \caption{NADPH producing and consuming fluxes in \textit{R. toruloides} with model iRhtoC. The model was optimized for NGAM minimization. 
    Glucose uptake and growth rate were constrained on the lowest rate.}
    \label{fig:iRhtoC_nm_NADPH_min}
\end{figure}
\begin{figure}[H]
    \centering
    \includegraphics[width=\linewidth]{iRhtoC_nm_NADPH1.png}
    \caption{NADPH producing and consuming fluxes in \textit{R. toruloides} with model iRhtoC. The model was optimized for NGAM minimization, 
    growth rate and glucose uptake were constrained to $0.08$ and $1.114$, respectively. 
    The results are the same when glucose uptake is constrained to $1.114, 1.648, 2.305$ or $3.1$.}
    \label{fig:iRhtoC_nm_NADPH1}
\end{figure}



\textbf{Rt\_IFO0880}

Model Rt\_IFO0880 optimized for NGAM minimisation predicts same 
exchage and intracellular fluxes as before. But production of NADPH differs - around $90\%$ is produced by homoserine dehydrogenase 
on rates $0.08$ and $0.17$ $(1/h)$ and on other rates by alcohol dehydrogenase (figures \ref{fig:IFO0880_nm_NADPH} and \ref{fig:IFO0880_nm_NADPH1}). 

\begin{figure}[H]
    \centering
    \includegraphics[width=\linewidth]{IFO0880_nm_NADPH.png}
    \caption{NADPH producing and consuming fluxes in \textit{R. toruloides} with model Rt\_IFO0880. The model was optimized for NGAM minimization. 
    Glucose uptake and growth rate were constrained on the lowest rate. The results are the same when glucose uptake is constrained to $1.648$ or $3.1$.}
    \label{fig:IFO0880_nm_NADPH}
\end{figure}
\begin{figure}[H]
    \centering
    \includegraphics[width=\linewidth]{IFO0880_nm_NADPH1.png}
    \caption{NADPH producing and consuming fluxes in \textit{R. toruloides} with model Rt\_IFO0880. The model was optimized for NGAM minimization, 
    growth rate and glucose uptake were constrained to $0.0$8 and $1.114$, respectively. The results are similar when glucose uptake is constrained to $2.305$.}
    \label{fig:IFO0880_nm_NADPH1}
\end{figure}


\textbf{Rt\_IFO0880\_LEBp2023}

Rt\_IFO0880\_LEBp2023 optimized for NGAM minimisation predicts 
very similar exchange and intracellular fluxes as before. 
The production of NADPH differs - on rates $0.08, 0.12$ and $0.17$ $(1/h)$ use of homoserine dehydrogenase is predicted instead of alcohol dehydrogenase (\ref{fig:Rt_IFO0880_LEBp2023_nm_NADPH1}). 
On lowest rate aldehyde dehdrogenase and on highest rate alcohol dehydrogenase is predicted, which is the same as previously with biomass 
maximisation as an objective function.  
\begin{figure}[H]
    \centering
    \includegraphics[width=\linewidth]{Rt_IFO0880_LEBp2023_nm_NADPH1.png}
    \caption{NADPH producing and consuming fluxes in \textit{R. toruloides} with model Rt\_IFO0880\_LEBp2023. The model was optimized for NGAM minimization, 
    growth rate and glucose uptake were constrained to $0.0$8 and $1.114$, respectively. The results are very similar when glucose uptake is constrained to $1.648$ or $2.305$.}
    \label{fig:Rt_IFO0880_LEBp2023_nm_NADPH1}
\end{figure}

For all simulations, the exchange fluxes are the same as with biomass maximisation as objective function. Also, fluxes of the intracellular
metabolites visualized in figures did not change. Meaning that the models that predicted phosphoketolase or ACL in previous simulations, 
predicted it again. For models besides rhto-GEM the production of NADPH differs with NGAM as an objective function.
With model iRhtoC, most of the NADPH is still produced by oxPPP, but the remaining $6\%$ on all growth rates besides the lowest, 
is produced by isocitrate dehydrogenase when optimized for NGAM minimisation instead of malic enzyme. Model Rt\_IFO0880  
predicts that on rates $0.08$ and $0.17$ $(1/h)$ NADPH is primarily produced by homoserine dehydrogenase instead of homoserine dehydrogenase.
Rt\_IFO0880\_LEBp2023 also predicts the use of homoserine dehydrogenase instead of alcohol dehydrogenase on rates $0.08, 0.12$ and $0.17$ $(1/h)$.




% % Discussion
% Still, the autohor of the enhanced model of Rt\_IFO0880 named Rt\_IFO0880\_LEBp2023, has pointed out that the improvements that resulted in the Rt\_IFO0880\_LEBp2023 
% model cover only a small portion of the metabolism of \textit{R. torluoides}. "The differences are possibly the "tip of the iceberg" of many others and
% the differences between the existing GEMs of \textit{R. toruloides} are significant and may lead to different understandings of
% this yeast's physiology if used alone" \cite{DeBiaggi2023}.

%  https://www.ncbi.nlm.nih.gov/pmc/articles/PMC9560524/ an article were dif obj funcs were compared




