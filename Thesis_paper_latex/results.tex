\chapter{Results}

% Characterizing flux differences 
% (Key figures: e.g. scatter plots, simulated flux maps, cofactor pie plots)


% Results Section 1. a figure panel with 
% - A: brief metabolic map; 
% - B: exchange fluxes;
% - C: main intracellular fluxes. Different GRs on x axis


\textit{Rhodotorula toruloides} can naturally accumulate high amounts of lipids, but the 
metabolic mechanism that make this possible and differentiate \textit{R. toruloides} from non-oleaginous yeast are not fully understood.
Biosynthesis of the main lipid precursors acetyl-CoA and NADPH takes place in the central carbon metabolism. A better understanding of which 
metabolic pathways are used in production of these precursors and thus contribute to lipid accumulation, would aid in designing better 
metabolic engineering strategies for increasing lipid production.

Genome-scale metabolic models contain all known biochemical reactions of the specific organism and allow the calculation of
metabolic fluxes, which represent the activity of metabolic pathways under specified conditions. 
This makes GEMs important tool for studying metabolism, but it is important that the predictive power of the GEM is
good. For \textit{R. toruloides} several genome-scale metabolic models are available, but so far comprehensive overview of 
simulations focused on central carbon metabolism with these models has not been presented.
This work will help in the future to design \textit{R. toruloides} as the leading microbial cell factory for production of microbial oils.


\section{Metabolic map} % Add metabolic map of central carbon metabolism in Rt


\section{Intracellular fluxes and supply of NADPH} %
% Describe the fluxes in central carbon metabolism of R. toruloides;
%% Explain the preference towards XPK or ACL pathway for producing acetyl-CoA

% Extract cofactor balances and analyze what makes model to prefer one or the other.
% Explain where the NADPH is produced  

% For plotting, the enzymes that had fluxes lower than 2% were added to the 'Others' section


% Also cofactor balances of NADH and ATP were plotted.
% * For the model rhto-GEM, there is no difference between any cofactor balances on different growth rates nor objective function
% * iRhtoC - same as with rhto-GEM
% * IFO0880 - differences between the objective functions on the lowest growth rate: biomass max predicts alcohol dehydrogenase as the main source of NADPH (a loop??), NGAM min predicts Homoserine dehydrogenase
%  on the highest GR both obj functions predict alcohol dehydrogenase as the main source, consuming fluxes differ slightly on both GRs - but for us it's not important?
% NADH and ATP fluxes don't differ
% * IFO0880_jsb - NADPH: on lowest GR both obj functions predict fairly similarly that aldehyde dehydrogenase is the main source of NADPH
% on highest GR biomass max as obj predicts alcohol dehydrogenase and NGAM min predicts homoserine dehydrogenase
% NADH fluxes don't differ on different GR rates but they differ between biomass max and NGAM min as obj:
%  biomass max - predicts several enzymes, NGAM min redicts glyceraldehyde-phosphate dehydrogenase as the main source of NADH
% ATP don't differ between GR but between objective functions: biomass max predicts ATP synthase as the main source and NGAM min ADP/ATP transporter

% Differences between models:
% 

% \textbf{Biomass maximisation as objective function}

% % Solution fluxes of rhto-GEM showed that NADPH is recycled via ... . 
% Production and consumption of NADPH with the different models is shown in figures below.


Four genome-scale metabolic models of \textit{Rhodotorula toruloides} - rhto-GEM, iRhto1108, Rt\_IFO0880 and Rt\_IFO0880\_LEBp2023 - 
were used for simulating this yeast's metabolic flux distribution using flux balance analysis. 
Different objective functions, maximisation of biomass and minimisation
of NGAM, were tested under carbon limitation. 
To see the flux patterns over different biomass growth rates, the 
solutions were constrained over five glucose uptakes - $0.476, 1.114, 1.648, 2.305, 3.1  (mmol/gDW/h)$ - that were measured in lab experiments.

Fluxes through notable phosphoketolase and ATP-citrate lyase pathways, 
that are known to be important for production of lipid precursors in olegeanous yeast, were in the center of interest and are distinguished in intracellular 
flux figures from other enzymes using dashed line.
What is more, lipid synthesis demands high amounts of NADPH, but only a few enzymes can generate it. 
Pentose phosphate pathway and malic enzyme have been proposed as the main candidate enzymes for recycling NADPH in \textit{R. toruloides} 
\cite{Ratledge2014}. Main intracellular fluxes of central carbon metabolism and production and consumption of NADPH were compared between the simulations 
of four models. 

For all simulations exchange fluxes were also observed
for evaluating model prediction compared to experimental data. For easier comparison of model predicted exchange fluxes with experimentally measured 
exchange fluxes,
the experimental data is illustrated on a graph \ref{experimental_ex_fluxes}.

\begin{figure}[h!]
    \centering
    \includegraphics[width=0.5\linewidth]{experimental_ex_fluxes.png}
    \caption{Experimental exchange fluxes in \textit{R. toruloides} strain IFO0880 over five growth rates.}
    \label{experimental_ex_fluxes}
\end{figure}


\subsection{Biomass maximisation as objective function}

In first simulations biomass maximisation was used as objective function and the calculations were made under carbon limitation constraining glucose 
uptake. Simulation results with all four models are shown in the figures below. All figures show the fluxes over growth rates from 0.05 - 0.25 ($1/h$).
Exchange fluxes graphs show the exchange (uptake or secretion) rate of glucose, oxygen, ammonium, sulphate, phosphate and carbon dioxide.
Intracellular fluxes graphs show the fluxes of NGAM, glucose uptake, glucose 6-phosphate dehydrogenase (oxPPP), transketolase 1, transaldolase, 
transketolase 2, 
fructose-biphosphate aldolase, pyruvate decarboxylase, pyruvate dehydrogenase, phosphoketolase and ATP-citrate lyase.

These simulations, with biomass maximisation as objective function under carbon limitation, showed that the models prefer different pathways for 
production of acetyl-CoA. The models rhto-GEM and Rt\_IFO0880 produced acetyl-CoA through phosphoketolase, whereas models iRhtoC and the modified model 
of Rt\_IFO0880 (Rt\_IFO0880\_LEBp2023) produced acetyl-CoA using ACL. NADPH sources also differ. Each simulatio figures and results are described in more detail below.



\textbf{rhto-GEM}

The simulation results with rhto-GEM model for exchange and intracellular fluxes are shown in figure \ref{rhto-GEM biomass max} and fluxes of NADPH 
producing and consuming enzymes are shown in \ref{fig:rhtoGEM_bm_NADPH}. Compared to the values obtained in lab experiments the model predicts higher exchange fluxes 
on a given growth rate. This is also a reason why it was chosen to constrain the simulations with lab glucose uptake data and model simulated growth rates, because 
otherwise the model predicted exchange fluxes would be even higher. This result was expected as GEMs are known to overestimate fluxes per growth rate.

Model rhto-GEM predicts the use of phosphoketolase on all growth rates and no use of ATP-citrate lyase. For NADPH production and consumption there are no differences 
between lowest and highest rate. In both cases the model predicts that over 90\% of the NADPH is produced by glucose 6-phosphate dehydrogenase (oxPPP) and
phosphogluconate dehydrogenase. 6.3\% is produced by methylenetetrahydrofolate dehydrogenase.



\begin{figure}[h!]
    \centering
    \includegraphics[width=0.9\linewidth]{rhtoGEM_biomass_max.png}
    \caption{Exchange and intracellular fluxes in \textit{R. toruloides} with model rhto-GEM optimized for biomass maximization 
    and constrained over five glucose uptake rates.}
    \label{rhto-GEM biomass max}
\end{figure}

\begin{figure}[h!]
    \centering
    \begin{subfigure}[h!]{0.49\textwidth}
        \centering
        \includegraphics[width=\textwidth]{rhtoGEM_bm_NADPH.png}
        % \caption{}
    \end{subfigure}
    \hfill
    \begin{subfigure}[h!]{0.49\textwidth}
        \centering
        \includegraphics[width=\textwidth]{rhtoGEM_bm_NADPH_max.png}
        % \caption{}
    \end{subfigure}
    \caption{NADPH producing and consuming fluxes in \textit{R. toruloides} with model rhto-GEM optimized for biomass maximization. 
    Glucose uptake was constrained on the lowest (left) and highest (right) rate.}
    \label{fig:rhtoGEM_bm_NADPH}
\end{figure}




\textbf{iRhtoC}

The results with iRhtoC are shown in figures \ref{iRhtoC_biomass_max} and \ref{fig:iRhtoC_bm_NADPH}. 
This model predicts very similar exchange fluxes as the previous model. But predicted intracellular fluxes differ - iRhtoC predicts the use of 
ATP-citrate lyase instead of phosphoketolase on all growth rates. 
With this model there is also no difference in NADPH production and consumption on highest and lowest rates. 
Almost 90\% of NADPH is produced by glucose 6-phosphate dehydrogenase (oxPPP) and
phosphogluconate dehydrogenase and 6.4\% by malic enzyme. %Predicted consuming fluxes differ from rhto-GEM predictions by the use of malic enzyme.


\begin{figure}[h!]
    \centering
    \includegraphics[width=0.9\linewidth]{iRhtoC_biomass_max.png}
    \caption{Exchange and intracellular fluxes in \textit{R. toruloides} with model iRhtoC optimized for biomass maximization 
    and constrained over five glucose uptake rates.}
    \label{iRhtoC_biomass_max}
\end{figure}

\begin{figure}[h!]
    \centering
    \begin{subfigure}[h!]{0.49\textwidth}
        \centering
        \includegraphics[width=\textwidth]{iRhtoC_bm_NADPH_min.png}
        % \caption{}
    \end{subfigure}
    \hfill
    \begin{subfigure}[h!]{0.49\textwidth}
        \centering
        \includegraphics[width=\textwidth]{iRhtoC_bm_NADPH_max.png}
        % \caption{}
    \end{subfigure}
    \caption{NADPH producing and consuming fluxes in \textit{R. toruloides} with model iRhtoC. The model was optimized for biomass maximization. 
    Glucose uptake was constrained on the lowest (left) and highest (right) rate.}
    \label{fig:iRhtoC_bm_NADPH}
\end{figure}


\textbf{Rt\_IFO0880}

Model Rt\_IFO0880 fluxes are shown in \ref{fig:IFO0880_biomass_max} and \ref{fig:IFO0880_biomass_max_NADPH_max}.
Exchange fluxes are similar with previous models. On all growth rates the use of phosphoketolase is predicted and no use of ACL. 
(This model has two phosphoketolases, fructose-6-phosphate phosphoketolase (FPK) and xylulose-5-phosphate phosphoketolase (XPK), but as models rhto-GEM and iRhtoC
have one phosphoketolase, FPK and XPK fluxes have been summed together for easier comparison with other models.)
In NADPH production and consumption on highest and lowest rates there are not significant differences. 
Around 90\% of NADPH is produced by alcohol dehydrogenase. %(That indicates unrealistic loops?)



\begin{figure}[h!]
    \centering
    \includegraphics[width=0.9\linewidth]{IFO0880_biomass_max.png}
    \caption{Exchange and intracellular fluxes in \textit{R. toruloides} with model Rt\_IFO0880 optimized for biomass maximization 
    and constrained over five glucose uptake rates.}
    \label{fig:IFO0880_biomass_max}
\end{figure}

\begin{figure}[h!]
    \centering
    \begin{subfigure}[h!]{0.49\textwidth}
        \centering
        \includegraphics[width=\textwidth]{IFO0880_biomass_max_NADPH.png}
        % \caption{}
    \end{subfigure}
    \hfill
    \begin{subfigure}[h!]{0.49\textwidth}
        \centering
        \includegraphics[width=\textwidth]{IFO0880_biomass_max_NADPH_max.png}
        % \caption{}
    \end{subfigure}
    \caption{NADPH producing and consuming fluxes in \textit{R. toruloides} with model Rt\_IFO0880. The model was optimized for biomass maximization. 
     Glucose uptake was constrained on the lowest (left) and highest (right) rate.}
    \label{fig:IFO0880_biomass_max_NADPH_max}
\end{figure}


\textbf{Rt\_IFO0880\_LEBp2023}

In figures \ref{fig:Rt_IFO0880_LEBp2023_biomass_max} and \ref{fig:Rt_IFO0880_LEBp2023_biomass_max_NADPH_max} 
simulation predictions with model Rt\_IFO0880\_LEBp2023 are shown.
Exchange fluxes align with other models. On all growth rates the use of ACL is predicted instead of phosphoketolase. 
(As this model is an updated version of Rt\_IFO0880, it also has two phosphoketolases - FPK and XPK, which fluxes
have been summed together and represented as phosphoketolase.) 
On lowest growth rate NADPH is predicted to be produced mainly by aldehyde dehydrogenase and on highest rate by alcohol dehydrogenase.
% (That indicates unrealistic loops?)


\begin{figure}[h!]
    \centering
    \includegraphics[width=0.9\linewidth]{Rt_IFO0880_LEBp2023_biomass_max.png}
    \caption{Exchange and intracellular fluxes in \textit{R. toruloides} with model Rt\_IFO0880\_LEBp2023 optimized for biomass maximization 
    and constrained over five glucose uptake rates.}
    \label{fig:Rt_IFO0880_LEBp2023_biomass_max}
\end{figure}

\begin{figure}[h!]
    \centering
    \begin{subfigure}[h!]{0.49\textwidth}
        \centering
        \includegraphics[width=\textwidth]{Rt_IFO0880_LEBp2023_biomass_max_NADPH.png}
        % \caption{}
    \end{subfigure}
    \hfill
    \begin{subfigure}[h!]{0.49\textwidth}
        \centering
        \includegraphics[width=\textwidth]{Rt_IFO0880_LEBp2023_biomass_max_NADPH_max.png}
        % \caption{}
    \end{subfigure}
    \caption{NADPH producing and consuming fluxes in \textit{R. toruloides} with model Rt\_IFO0880\_LEBp2023. The model was optimized for biomass maximization. 
    Glucose uptake was constrained on the lowest (left) and highest (right) rate.}
    \label{fig:Rt_IFO0880_LEBp2023_biomass_max_NADPH_max}
\end{figure}



\subsection{NGAM minimisation as objective function}

NGAM minimisation as objective function was investigated to see whether the use of different objective function gives distinct results. 
Again the simulations were made under carbon limitation over five glucose uptakes. This objective function also needed constraints 
on biomass growth rate because otherwise the simulation chooses zero as its flux. Growth rate was constrained to the values that each model
predicted in the simulations optimized for biomass maximisation, because experimental growth rates together with experimental glucose 
uptakes were infeasible for the models. 

For all simulations, the exchange fluxes are the same as with biomass maximisation as objective function. The models that predicted phosphoketolase 
in previous simulations, predicted it again. Same is about prediction of ACL. For some models the production of NADPH differs with this objective function.

\textbf{rhto-GEM}

Model rhto-GEM optimized for NGAM minimisation (see figures \ref{fig:rhtoGEM_NGAM_min} and \ref{fig:rhtoGEM_nm_NADPH}) predicts similar fluxes as with 
previous objective function. 

\begin{figure}[h!]
    \centering
    \includegraphics[width=0.9\linewidth]{rhtoGEM_NGAM_min.png}
    \caption{Exchange and intracellular fluxes in \textit{R. toruloides} with model rhto-GEM optimized for NGAM minimization 
    and constrained over five growth and glucose uptake rates.}
    \label{fig:rhtoGEM_NGAM_min}
\end{figure}

\begin{figure}[h!]
    \centering
    \begin{subfigure}[h!]{0.49\textwidth}
        \centering
        \includegraphics[width=\textwidth]{rhtoGEM_nm_NADPH.png}
        % \caption{}
    \end{subfigure}
    \hfill
    \begin{subfigure}[h!]{0.49\textwidth}
        \centering
        \includegraphics[width=\textwidth]{rhtoGEM_nm_NADPH_max.png}
        % \caption{}
    \end{subfigure}
    \caption{NADPH producing and consuming fluxes in \textit{R. toruloides} with model rhto-GEM. The model was optimized for NGAM minimization. 
    Glucose uptake with growth rate were constrained on the lowest (left) and highest (right) rate.}
    \label{fig:rhtoGEM_nm_NADPH}
\end{figure}



\textbf{iRhtoC}

iRhtoC optimized for NGAM minimisation (figures \ref{fig:iRhtoC_NGAM_min} and \ref{fig:iRhtoC_nm_NADPH}) also predicts similar fluxes as with 
previous objective function. 

\begin{figure}[h!]
    \centering
    \includegraphics[width=0.9\linewidth]{iRhtoC_NGAM_min.png}
    \caption{Exchange and intracellular fluxes in \textit{R. toruloides} with model iRhtoC optimized for NGAM minimization 
    and constrained over five growth and glucose uptake rates.}
    \label{fig:iRhtoC_NGAM_min}

\end{figure}


\begin{figure}[h!]
    \centering
    \begin{subfigure}[h!]{0.49\textwidth}
        \centering
        \includegraphics[width=\textwidth]{iRhtoC_nm_NADPH_min.png}
        % \caption{}
    \end{subfigure}
    \hfill
    \begin{subfigure}[h!]{0.49\textwidth}
        \centering
        \includegraphics[width=\textwidth]{iRhtoC_nm_NADPH_max.png}
        % \caption{}
    \end{subfigure}
    \caption{NADPH producing and consuming fluxes in \textit{R. toruloides} with model iRhtoC. The model was optimized for NGAM minimization. 
    Glucose uptake with growth rate were constrained on the lowest (left) and highest (right) rate.}
    \label{fig:iRhtoC_nm_NADPH}
\end{figure}



\textbf{Rt\_IFO0880}

Model Rt\_IFO0880 optimized for NGAM minimisation (figures \ref{fig:IFO0880_NGAM_min} and \ref{fig:IFO0880_nm_NADPH}) predicts similar intracellular
fluxes as before, but NADPH production differs. On the lowest growth rate it is predicted that most of NADPH is produced by homoserine dehydrogenase 
and on highest rate by alcohol dehydrogenase. 


\begin{figure}[h!]
    \centering
    \includegraphics[width=0.9\linewidth]{IFO0880_NGAM_min.png}
    \caption{Exchange and intracellular fluxes in \textit{R. toruloides} with model Rt\_IFO0880 optimized for NGAM minimization 
    and constrained over five growth and glucose uptake rates.}
    \label{fig:IFO0880_NGAM_min}
\end{figure}

\begin{figure}[h!]
    \centering
    \begin{subfigure}[h!]{0.49\textwidth}
        \centering
        \includegraphics[width=\textwidth]{IFO0880_nm_NADPH.png}
        % \caption{}
    \end{subfigure}
    \hfill
    \begin{subfigure}[h!]{0.49\textwidth}
        \centering
        \includegraphics[width=\textwidth]{IFO0880_nm_NADPH_max.png}
        % \caption{}
    \end{subfigure}
    \caption{NADPH producing and consuming fluxes in \textit{R. toruloides} with model Rt\_IFO0880. The model was optimized for NGAM minimization. 
    Glucose uptake with growth rate were constrained on the lowest (left) and highest (right) rate.}
    \label{fig:IFO0880_nm_NADPH}
\end{figure}



\textbf{Rt\_IFO0880\_LEBp2023}

Rt\_IFO0880\_LEBp2023 optimized for NGAM minimisation (figures \ref{fig:Rt_IFO0880_LEBp2023_NGAM_min} and \ref{fig:Rt_IFO0880_LEBp2023_nm_NADPH}) also predicts 
similar intracellular fluxes as before, but the production of NADPH differs. On the lowest growth rate it is predicted that most of NADPH is produced by 
aldehyde dehydrogenase and on highest rate by homoserine dehydrogenase. 

\begin{figure}[h!]
    \centering
    \includegraphics[width=0.9\linewidth]{Rt_IFO0880_LEBp2023_NGAM_min.png}
    \caption{Exchange and intracellular fluxes in \textit{R. toruloides} with model Rt\_IFO0880\_LEBp2023 optimized for NGAM minimization 
    and constrained over five growth and glucose uptake rates.}
    \label{fig:Rt_IFO0880_LEBp2023_NGAM_min}
\end{figure}

\begin{figure}[h!]
    \centering
    \begin{subfigure}[h!]{0.49\textwidth}
        \centering
        \includegraphics[width=\textwidth]{Rt_IFO0880_LEBp2023_nm_NADPH.png}
        % \caption{}
    \end{subfigure}
    \hfill
    \begin{subfigure}[h!]{0.49\textwidth}
        \centering
        \includegraphics[width=\textwidth]{Rt_IFO0880_LEBp2023_nm_NADPH_max.png}
        % \caption{}
    \end{subfigure}
    \caption{NADPH producing and consuming fluxes in \textit{R. toruloides} with model Rt\_IFO0880\_LEBp2023. The model was optimized for NGAM minimization. 
    Glucose uptake with growth rate were constrained on the lowest (left) and highest (right) rate.}
    \label{fig:Rt_IFO0880_LEBp2023_nm_NADPH}
\end{figure}







