\chapter*{Introduction}
\phantomsection
\addcontentsline{toc}{chapter}{Introduction}

% Sissejuhatuses esitatakse lühidalt (1-2 lk) probleemi tutvustus ja teema valiku põhjendus, töö probleem
% ja eesmärk. Tuuakse välja teema aktuaalsus ja uudsus.

% Introduction is going to be still about how FBA in R. toruloides has been analyzed on a rather general level, but to aid metabolic 
% engineering, we need to understand it more. Then you introduce different models, describe what 
% they did, mention that in S. cerevisiae very detailed FBA studies have been carried out, so more 
% specific FBA analysis in R. toruloides is necessary… That is why most of the writing is good to 
% do at the end when all is more clear.


% Hea lause
% \textit{R. toruloides} has been target of significant research efforts
% including genome (re)sequencing, functional genomics analyses, differential omics characterization, determination of macromolecular composition, and growth
% kinetics in a continuous culture. These experiments have ushered an improved understanding of \textit{R. toruloides}
% metabolism and provided the basis for the reconstruction of a metabolic
% model with genome-wide coverage. \cite{Dinh2019}


% ACL has not been found to be present in non-oleaginous yeasts. \cite{Vorapreeda2012}