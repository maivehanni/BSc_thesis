\chapter{Conclusion}
% I think that the conclusions will also cover things like, which was the best objective function and models’
%  performance over different growth rates. So leave your thoughts about the conclusion open:)

In this work four genome-scale metabolic models of \textit{Rhodotorula torluoides} were compared and different objective functions, including maximization of biomass and NGAM minimization, 
were tested. The fluxes through phosphoketolase and ACL pathways, that are found to be important for lipid production in all four models, were explored. 
This study gave more insight about how these model predict central carbon metabolism of \textit{R. torluoides} than \cite{DeBiaggi2023} that was based on phytoenes simulations. % Uuri lahemalt nende kohta

% In the Conclusions, we must list all 
% these discoveries, including which objective function performed the best and which main pathways were used (what you already have listed in Benchling notebook).

The model Rt\_IFO0880\_LEBp2023 was found to predict \textit{Rhodotorula torluoides} intracellular fluxes of lipogenesis precursors most accurately.
This model has cytosolic malate dehydrogenase and thus is able to predict the flux of ACL, which from omics studies has been found to be present in \textit{R. torluoides}.

Still, the autohor of the enhanced model of Rt\_IFO0880 named Rt\_IFO0880\_LEBp2023, has pointed out that the improvements that resulted in the Rt\_IFO0880\_LEBp2023 
model cover only a small portion of the metabolism of \textit{R. torluoides}. "The differences are possibly the "tip of the iceberg" of many others and
the differences between the existing GEMs of \textit{R. toruloides} are significant and may lead to different understandings of
this yeast's physiology if used alone" \cite{DeBiaggi2023}.

For better prediction of \textit{R. toruloides} phenotype there is a need for a consensus GEM.

% In the Conclusions, we must list all these discoveries, including which 
% objective function performed the best and which main pathways were used (what you already have listed in Benchling notebook).