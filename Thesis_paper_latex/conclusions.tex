\chapter{Conclusion}
% I think that the conclusions will also cover things like, which was the best objective function and models’
%  performance over different growth rates. So leave your thoughts about the conclusion open:)

In this work four genome-scale metabolic models of \textit{Rhodotorula torluoides} were compared and two objective functions, maximization of biomass and NGAM minimization, 
were compared. The fluxes through phosphoketolase and ACL pathways, that are found to be important for lipid production in all four models, were explored. 
This study gave more insight about how these model predict central carbon metabolism of \textit{R. torluoides} than \cite{DeBiaggi2023} that was based on phytoenes simulations. % Uuri lahemalt nende kohta

% In the Conclusions, we must list all 
% these discoveries, including which objective function performed the best and which main pathways were used (what you already have listed in Benchling notebook).

%This paragraph will require rewriting.

% Please, think more about conclusions.

% Firstly, it is not yet to be clear if jsb model is “more accurate” 
% because our next study will only tell:) So rephrase.

% But start by defining what your results gave - what looks different


% The model Rt\_IFO0880\_LEBp2023 was found to predict \textit{Rhodotorula toruloides} intracellular fluxes of lipogenesis precursors most accurately.
% This model has cytosolic malate dehydrogenase and thus is able to predict the flux of ACL, which from omics studies has been found to be present in \textit{R. torluoides}.

For better prediction of \textit{R. toruloides} phenotype there is a need for a consensus GEM.

% In the Conclusions, we must list all these discoveries, including which 
% objective function performed the best and which main pathways were used (what you already have listed in Benchling notebook).