\chapter{Conclusion}
% I think that the conclusions will also cover things like, which was the best objective function and models’
%  performance over different growth rates. So leave your thoughts about the conclusion open:)

In this work, four genome-scale metabolic models of \textit{Rhodotorula torluoides} were compared during simulations with glucose as a sole carbon substrate uptake, using two different objective functions - maximisation of biomass and minimisation of NGAM. The fluxes through phosphoketolase and ACL pathways, that are found to be important for lipid production, were explored. Results demonstrated clear difference in model predictions in regard to the use 
of one (models that predicted XPK: rhto-GEM, Rt\_IFO0880) or the other pathway (models that predicted ACL: iRhtoC and Rt\_IFO0880\_LEBp2023), respectively, and this did not change with different objective functions.

The consuming fluxes of NADPH also differed between the models. Two models (rhto-GEM and iRhtoC) predicted that most of the NADPH is produced by the oxidative part of pentose phosphate pathway and Rt\_IFO0880-based models predicted NADPH production mainly by alcohol or aldehyde dehydrogenase when model was optimized for biomass maximisation, whereas when optimized for NGAM minimisation, homoserine dehydrogenase was predicted on most growth rates instead. Only model iRhtoC predicted that malic enzyme was involved in the production of NADPH, but that corresponded to only 6\% of total produced NADPH.

% Please, think more about conclusions.

% But start by defining what your results gave - what looks different


% The model Rt\_IFO0880\_LEBp2023 was found to predict \textit{Rhodotorula toruloides} intracellular fluxes of lipogenesis precursors most accurately.
% This model has cytosolic malate dehydrogenase and thus is able to predict the flux of ACL, which from omics studies has been found to be present in \textit{R. torluoides}.
This study gave more insight about how the models rhto-GEM, iRhtoC, Rt\_IFO0880 and Rt\_IFO0880\_LEBp2023 predict central carbon metabolism of \textit{R. toruloides} than previously found in the literature. Still, as the models lead to different understanding of the production of lipid precursors acetyl-CoA and NADPH in \textit{R. toruloides}, there is a need for a consensus genome-scale metabolic model of \textit{R. toruloides} for obtaining more accurate predictions of \textit{R. toruloides} phenotype.

% In the Conclusions, we must list all these discoveries, including which 
% objective function performed the best and which main pathways were used (what you already have listed in Benchling notebook).