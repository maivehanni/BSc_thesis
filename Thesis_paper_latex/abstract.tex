\chapter*{Abstract}
\phantomsection
\addcontentsline{toc}{chapter}{Abstract}

% Kokkuvõte hõlmab töö olulisemad tulemused ja järeldused, töö eesmärgi täidetuse analüüs,
% ettepanekuid töö edasiarendamiseks ning edasiseks uurimistööks antud valdkonnas, maht kuni 1 lk.
% Kokkuvõte ei tohi sisaldada põhiosas käsitlemata seisukohti ja lahendusi.

% Annotatsioon annab lugejale ülevaate töö eesmärkidest, olulisematest käsitletud probleemidest ning
% tähtsamatest tulemustest ja järeldustest. Annotatsioon on töö lühitutvustus, mis ei selgita ega
% põhjenda, kuid kajastab piisavalt töö sisu. Annotatsioon esitatakse mahuga pool kuni üks A4 lehekülge
% lõputöö keeles, millele lisatakse eestikeelse töö puhul inglise keelne Abstract. Võõrkeelse töö puhul
% lisatakse eestikeelne annotatsioon (va ingliskeelse tasemeõppekava lõputöö korral). Erinevates keeltes
% annotatsioonid vormistatakse erinevatele lehekülgedele – annotatsiooni pealkiri on esimese taseme
% pealkiri, mis algab uuest leheküljest.

% A thesis must have an abstract or summary in both Estonian and English. An abstract provides the reader an overview of the objectives of the thesis, the key issues discussed and the most important findings and conclusions. An abstract is a brief introduction, which provides no explanations or arguments but reflects adequately the content of the thesis. An abstract must be half to one A4 page long. An abstract must not contain statements not discussed in the body text. Abstracts in different languages shall be written on separate pages. The version in the language of the thesis should come first.