\chapter*{Abstract}
\phantomsection
\addcontentsline{toc}{chapter}{Abstract}

% Kokkuvõte hõlmab töö olulisemad tulemused ja järeldused, töö eesmärgi täidetuse analüüs,
% ettepanekuid töö edasiarendamiseks ning edasiseks uurimistööks antud valdkonnas, maht kuni 1 lk.
% Kokkuvõte ei tohi sisaldada põhiosas käsitlemata seisukohti ja lahendusi.

% Annotatsioon annab lugejale ülevaate töö eesmärkidest, olulisematest käsitletud probleemidest ning
% tähtsamatest tulemustest ja järeldustest. Annotatsioon on töö lühitutvustus, mis ei selgita ega
% põhjenda, kuid kajastab piisavalt töö sisu. Annotatsioon esitatakse mahuga pool kuni üks A4 lehekülge
% lõputöö keeles, millele lisatakse eestikeelse töö puhul inglise keelne Abstract. Võõrkeelse töö puhul
% lisatakse eestikeelne annotatsioon (va ingliskeelse tasemeõppekava lõputöö korral). Erinevates keeltes
% annotatsioonid vormistatakse erinevatele lehekülgedele – annotatsiooni pealkiri on esimese taseme
% pealkiri, mis algab uuest leheküljest.

% A thesis must have an abstract or summary in both Estonian and English. An abstract provides the reader an overview of the objectives of the thesis, the key issues discussed and the most important findings and conclusions. An abstract is a brief introduction, which provides no explanations or arguments but reflects adequately the content of the thesis. An abstract must be half to one A4 page long. An abstract must not contain statements not discussed in the body text. Abstracts in different languages shall be written on separate pages. The version in the language of the thesis should come first.

The transition towards bioeconomy to reduce the dependence on fossil-based resources, necessitates innovative methods for producing chemicals and fuels from sustainable materials. The current production of biodiesel from oilseeds and waste oils is insufficient to meet the global demand of the biodiesel industry, highlighting the need for second-generation oleochemicals derived from non-edible sources. Microbial oils (SCOs), which utilize low-value waste streams and do not compete with the food sector, are a promising source of fatty acids for oleochemical production.
The non-conventional, oleaginous yeast \textit{Rhodotorula toruloides} is one of the most promising yeasts for bioproduction of oleochemicals.
Although its metabolic pathways that enable lipid production are defined, the way of enabling such a high lipid production is not fully understood.

Genome-scale metabolic models (GEMs) can be used to predict metabolic fluxes, enabling a greater understanding of cellular physiology, providing valuable information for metabolic engineering to develop better microbial factories. Several genome-scale metabolic models have been developed for \textit{Rhodotorula toruloides}, but a comprehensive overview of simulations focused on central carbon metabolism with these models has not yet been presented.
The objective of this thesis was to compare the predictions of central carbon metabolism focused on lipogenesis of four classical GEMs of \textit{R. toruloides} (rhto-GEM, iRhtoC, Rt\_IFO0880, and Rt\_IFO0880\_LEBp2023) using FBA optimized for biomass maximization and NGAM minimization over five increasing specific growth rates. The simulations showed that over the growth rates, investigated fluxes increased linearly. 

In regard to cofactor NADPH metabolism, the rhto-GEM and iRhtoC models predicted most of the NADPH to be produced through oxPPP, while the Rt\_IFO0880-based models predicted that most of the NADPH is produced by alcohol dehydrogenase, aldehyde dehydrogenase, or homoserine dehydrogenase, depending on the objective function and specific growth rate, indicating the need for a consensus genome-scale metabolic model of \textit{R. toruloides}.


