\chapter*{Annotatsioon}
\phantomsection
\addcontentsline{toc}{chapter}{Annotatsioon}

Kliimamuutuste vastu võitlemiseks ja vähendamaks sõltuvust fossiilsetest ressurssidest on paljud riigid üleminekul bio-põhisele majandusele. See üleminek nõuab innovaatilisi protsesse keemiliste ainete, materjalide ja kütuste jätkusuutlikuks tootmiseks. Praegune biodiisli tootmine õliseemnetest ja jäätmeõlidest ei ole piisav, et rahuldada globaalset nõudlust \cite{Koutinas2014}, rõhutades vajadust biokütuste järele, mis pärinevad mittesöödavatest allikatest.
Mikroobsed lipiidid (\textit{microbial oils}), tuntud ka kui ühe-raku õlid (\textit{single-cell oils}, SCOs), on paljutõotav allikas kemikaalide tootmiseks, kuna selleks saab kasutatada väheväärtuslikke jäätmeid, tänu millele ei konkureeri antud tootmine toiduainetööstusega.

Mitte-traditsiooniline õlirikas pärm \textit{Rhodotorula toruloides} on märkimisväärne mikroorganism bioproduktide tootmiseks, suutes akumuleerida väga suurtes kogustes lipiide.
Metaboolsed rajad, mis võimaldavad antud pärmis lipiidide tootmist, on üldiselt määratletud, kuid ainulaadsed metaboolsed omadused, mis võimaldavad kõrget lipiidide sünteesi, pole veel täielikult 
teada. Täpsemad teadmised vastavatest mehhanismidest võimaldaks tõhusamate rakuvabrikute bioinseneerimist.
Ülegenoomsed metabolismi mudelid (\textit{genome-scale metabolic models}, GEMs) võimaldavad raku ainevahetuse põhjalikku \textit{in silico} uurimist. Pärmi \textit{Rhodotorula toruloides} jaoks on sõltumatult välja töötatud mitu ülegenoomset mudelit, kuid puudub põhjalik ülevaade antud mudelitega ennustatud kesksest süsiniku metabolismist.

Käesoleva töö eesmärk oli võrrelda nelja mudeli (rhto-GEM, iRhtoC, Rt\_IFO0880 ja Rt\_IFO0880\_LEBp2023) ennustusi, keskendudes lipogeneesi prekursoreid, nimelt atsetüül-CoA-d ja kofaktor NADPH-d, tootvatele radadele. Metaboolsete voogude simuleerimiseks kasutati metaboolsete voogude analüüsi (\textit{flux balance analysis}, FBA). Raku fenotüüpi prognoositi viie spetsiifilise kasvukiiruse korral, kasutades eksperimentaalseid \textit{R. toruloides} pidevkultuuri (\textit{continuous cultivation}) andmeid. 

Simulatsioonidest selgus, et uuritud reaktsioonide vood (\textit{fluxes}) kasvasid lineaarselt koos kasvukiiruse tõusuga.  
Tulemused näitasid selget erinevust mudelite ennustustes atsetüül-CoA tootmiseks: mudelid rhto-GEM ja Rt\_IFO0880 ennustasid fosfoketolaasi rada ning mudelid iRhtoC ja Rt\_IFO0880\_LEBp2023 ennustasid ATP tsitraadi lüaasi (\textit{ATP-citrate lyase}) kasutamist. Ka kofaktor NADPH tootmise ennustused erinesid. Mudelid rhto-GEM ja iRhtoC ennustasid, et suurem osa NADPH-st toodetakse pentoosfosfaadi raja oksüdatiivses osas, samas kui Rt\_IFO0880-põhised mudelid ennustasid enamiku NADPH tootmist alkoholi dehüdrogenaasi, aldehüüd dehüdrogenaasi või homoseriin dehüdrogenaasi kaudu. Leitud tulemused näitavad vajadust konsensus \textit{R. toruloides} ülegenoomse metabolismi mudeli järele.

