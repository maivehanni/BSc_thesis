\chapter*{Abbreviations}
\phantomsection
\addcontentsline{toc}{chapter}{Abbreviations}

% Lühendite loetelus esitatakse lühendid tähestikulises järjekorras. Lühendite ning mõistete sõnastikku
% lisatakse töö põhitekstis kasutatud uued ning ka mitmetähenduslikud üldtuntud terminid. Olenemata
% lühendi lisamisest tabelisse, tuleb see töö põhiosa tekstis esmakordsel mainimisel alati lahti seletada.
% Vajadusel võib lühendite seletused esitada ka põhitekstis nende esmamainimisel.

% Tahestiku jrkr
\begin{acronym}
\acro{ACL}{ATP-citrate lyase}
\acro{acetyl-CoA}{acetyl-coenzyme A}
\acro{ATP}{adenosine triphosphate}
\acro{BiGG knowledge-base}{biochemical, genetic, and genomic knowledge-base}
\acro{C/N}{carbon-to-nitrogen ratio}
\acro{cMDH}{cytosolic malate dehydrogenase}
\acro{DNA}{deoxyribonucleic acid}
\acro{DW}{dry cellular weight}
\acro{FAEE}{fatty acid ethyl ester}
\acro{FAME}{fatty acid methyl ester}
\acro{FAS}{fatty acid synthase}
\acro{FBA}{flux balance analysis}
\acro{GAM reaction}{growth-associated maintenance reaction}
\acro{GEM}{genome-scale model}
\acro{ME}{malic enzyme}
\acro{NADPH}{nicotinamide adenine dinucleotide phosphate (reduced form)}
\acro{NGAM reaction}{non-growth-associated maintenance reaction}
\acro{oxPPP}{oxidative part of pentose phosphate pathway}
\acro{PPP}{pentose phosphate pathway}
\acro{RNA}{ribonucleic acid}
\acro{SBML}{Systems Biology Markup Language}
\acro{SCO}{single-cell oil}
\acro{TAG}{triacylglycerol}
\acro{TCA cycle}{tricarboxylic acid cycle (citric acid cycle)}
\end{acronym}
