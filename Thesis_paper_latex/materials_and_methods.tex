\chapter{Methods}

% Biomass reaction flux in $h^{-1}$ 
% Other fluxes in mmol $gDW^{-1} hr^{-1}1$
% COBRApy \cite{Ebrahim2013}
% FBA \cite{Orth2010}

% Perform flux balance analysis of R. toruloides at
% at least 5 different dilution rates (D 0.05 .. 0.2 h-1)
% under carbon limitation.
% Describe central carbon metabolic fluxes by different models through:
% - PP and XPK pathway
% - ACL and the TCA cycle
% Extract and analyze cofactor balances

Genome-scale metabolic models of \textit{R. toruloides} (JSON files) were obtained from supplemental
files from publications \cite{Tiukova2019, Dinh2019, Kim2021} and the fourth model \cite{DeBiaggi2023}, 
made by a group member, was obtained directly from him. 

With all these models, simulations were performed using the COBRApy
package (version 0.29.0) \cite{Ebrahim2013} in Python (version 3.11.4). 
Throughout the process, metabolic flux patterns were predicted using flux balance analysis \cite{Orth2010} from COBRApy package
with the Gurobi mathematical optimization solver (version 11.0.0, Gurobi Optimization Inc.).

All scripts are available in Github repository upon request sent to maive.hanni@gmail.com.


\section{Selecting experimental data}

In the simulations the experimental steady-state cultivation data of \textit{R. toruloides} strain IFO0880 in 1 L bioreactor was used.
The experiment was done by our group and the results have not yet been published. The cultivation process is described in \cite{Rekena2023}. 
All rates are expressed in $mmol/gDW/h$, and the biomass specific growth rate 
is expressed as ${1/h}$. The data is shown in the table \ref{table:LabData}.

\begin{table}[h]
    \centering
    \caption{Continuous cultivation results from lab experiments}
    \begin{tabular}{c|c|c|c|c}
        
            \textbf{Biomass growth rate} & \textbf{Glucose uptake} & \textbf{CO$_2$} & \textbf{O$_2$} & \textbf{Glycerol} \\ \hline
            0.049 & 0.476 & 1.171 & 1.083 & almost nothing \\ 
            0.100 & 1.114 & 2.521 & 2.521 & ~ \\ 
            0.151 & 1.648 & 3.854 & 3.851 & ~ \\ 
            0.203 & 2.305 & 5.834 & 4.352 & ~ \\ 
            0.25 & - & - & - & ~  \\ 
            0.301 & 3.1 & 7.415 & 6.327 & ~ \\ 
        \end{tabular}
    \label{table:LabData}
\end{table}


\section{Biomass integration to the model}

% Please, describe really briefly that we used default biomass composition in these models that corresponds to X, Y , Z lipid content 
% in biomass, measured in respective publications. To demonstrate that biomass composition was comparable in all models.



\section{Flux balance analysis and sampling of the solution space} 

% If there is time, would be nice to perform random sampling (like OptGPSampler in cobra) for the key FBA simulations in the thesis.

Growth phenotypes were obtained using FBA with several objective functions. First, before having lab data for glucose uptake, 
the objective function was set to glucose uptake maximisation and the biomass pseudoreaction, whose flux is equivalent to the growth 
rate ($h^{1/h}$), was constrained at five different rates (0.05, 0.10, 0.15, 0.2, 0.25, 0.30) (as glucose uptake flux is notated in the model 
with negative sign, setting the objective to glucose uptake maximisation corresponds to finding the solution space with minimal 
absolute value of glucose uptake flux). % Try changing the phrasing...
Model solutions were calculated on five different growth rates and Python's Pandas package was used for constructing 
dataframes of extracellular and the fluxes of central carbon metabolism enzymes. Matplotlib was used for data visualization.

Once lab data for steady-state cultivation of \textit{R. toruloides} strain IFO0880 was available, glucose uptake values 
on five growth rates were used in simulations for model constraining. On carbon limitation the objective
function was set to biomass maximisation. Again dataframes of the fluxes of interest were made and visualized.

Metabolic flux patterns were further investigated using the minimisation of non-growth associated maintenance (NGAM) as objective 
function with constraining both glucose uptake and growth rate to see if that changes the flux patterns. 
As the model overestimates glucose uptake need for a specific growth rate, 
it was not possible to constrain both glucose uptake and growth rate to the values obtained in lab - the solution was infeasible.
Because of that, in this simulation glucose uptake was constrained to the values obtained in lab, but growth rate was constrained to 
the growth rate values obtained in previous simulation when solution was optimized for maximum growth rate on specified glucose uptake. 

For comparison of cofactor balances between the models, pie plots were made to visualize the production and consumption of NADPH, NADH and ATP. % Maybe we won't include ATP bc it was the same  
Cofactor balances were simulated with biomass maximisation and NGAM minimisation 
as the objective function, as the lab data was available.

Finally, for some models the central carbon metabolism fluxes were visualized with Escher \cite{King2015}. On Escher platform it is possible
to create metabolic maps based on the reactions of a given GEM, which can then be fed with the simulated
flux distributions to have a better overview of used metabolic pathways.
