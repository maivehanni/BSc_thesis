\chapter{Methods}

% Biomass reaction flux in $h^{-1}$ 
% Other fluxes in mmol $gDW^{-1} hr^{-1}1$
% COBRApy \cite{Ebrahim2013}
% FBA \cite{Orth2010}

% Perform flux balance analysis of R. toruloides at
% at least 5 different dilution rates (D 0.05 .. 0.2 h-1)
% under carbon limitation.
% Describe central carbon metabolic fluxes by different models through:
% - PP and XPK pathway
% - ACL and the TCA cycle
% Extract and analyze cofactor balances
\section{Models}

Genome-scale metabolic models of \textit{R. toruloides} (JSON files) rhto-GEM, iRhtoC, Rt\_IFO0880 were obtained from supplemental
files from publications \cite{Tiukova2019, Dinh2019, Kim2021} and the fourth model Rt\_IFO0880\_LEBp2023 \cite{DeBiaggi2023}, 
made by a group member, was obtained directly from him. 

All scripts are available in Github repository (\url{https://github.com/maivehanni/BSc_thesis}) upon request sent to maive.hanni@gmail.com.


\section{Selecting experimental data}

In the simulations the experimental steady-state cultivation data of \textit{R. toruloides} strain IFO0880 in 1 L bioreactor was used.
The experiment was done by our group and the results have not yet been published. The cultivation process is described in \cite{Rekena2023}. 
All rates are expressed in \unit{mmol/gDW/h}, and the biomass specific growth rate 
is expressed as \unit{1/h}. The data is shown in the table \ref{table:LabData}.

\begin{table}[h]
    \centering
    \caption{Continuous cultivation results from lab experiments}
    \begin{tabular}{c|c|c|c|c}
        
            \textbf{Biomass growth rate \unit{1/h}} & \textbf{Glucose uptake \unit{mmol/gDW/h}} & \textbf{CO$_2$ \unit{mmol/gDW/h}} & \textbf{O$_2$ \unit{mmol/gDW/h}} & \textbf{Glycerol \unit{mmol/gDW/h}} \\ \hline
            0.049 & 0.476 & 1.171 & 1.083 & almost nothing \\ 
            0.100 & 1.114 & 2.521 & 2.521 & ~ \\ 
            0.151 & 1.648 & 3.854 & 3.851 & ~ \\ 
            0.203 & 2.305 & 5.834 & 4.352 & ~ \\ 
            0.25 & - & - & - & ~  \\ 
            0.301 & 3.1 & 7.415 & 6.327 & ~ \\ 
        \end{tabular}
    \label{table:LabData}
\end{table}


\section{Biomass integration to the model}

% Please, describe really briefly that we used default biomass composition in these models that corresponds to X, Y , Z lipid content 
% in biomass, measured in respective publications. To demonstrate that biomass composition was comparable in all models.



\section{Flux balance analysis and sampling of the solution space} 

% If there is time, would be nice to perform random sampling (like OptGPSampler in cobra) for the key FBA simulations in the thesis.
Model simulations were performed using the COBRApy
package (version 0.29.0) \cite{Ebrahim2013} in Python (version 3.11.4) with all four models. 
Throughout the process, metabolic flux patterns were predicted using flux balance analysis \cite{Orth2010} from COBRApy package with the Gurobi mathematical optimization solver (version 11.0.0, Gurobi Optimization Inc.). 

The following functions from COBRApy package were used:
\verb|read_sbml_model()| for importing the metabolic model in SBML format; model.objective for defining the objective function, \verb|model.reactions.get_by_id().bounds| for assigning bounds; \verb|model.optimize()| for calculating the solution (default is the maximization of the objective function and the minimization is achieved by using \verb|model.optimize('minimize')|); \verb|loopless_solution()| for obtaining a new flux distribution, where the sum of absolute non-exchange fluxes is minimized (\verb|loopless_solution()| is based on a previously obtained reference flux distribution with the function \verb|model.optimize()|); \verb|model.reactions.reaction.name| was used for obtaining the names of reactions instead of their IDs.
Calculated fluxes were stored in a Pandas dataframe and the fluxes of interest were further visualized using Matplotlib package.

In silico metabolic flux patterns were obtained using FBA with several objective functions. 
Model simulations were carried out on five different growth rates.
Firstly, glucose uptake values 
over five growth rates, were used in simulations for model constraining. On carbon limitation the objective function was set to biomass maximisation. Dataframes of the fluxes of interest were made and visualized.

To see whether different objctive function changes the flux patterns, metabolic flux patterns were further investigated using the minimisation of non-growth associated maintenance (NGAM) reaction as an objective function. Both the glucose uptake and growth rate were constrained. 
As the model overestimates glucose uptake need for a specific growth rate, it was not possible to constrain both glucose uptake and growth rate to the values obtained in lab - the solution was infeasible. Because of that, in this simulation glucose uptake was constrained to the values obtained in lab, but growth rate was constrained to the growth rate values obtained in previous simulation when model was optimized for biomass maximisation. On average, model overestimation of glucose uptake per growth rate was around 25\% of experimental glucose uptake rate and it was considered a reasonable assumption in this case.

For comparison of cofactor balances between the models, pie plots were made using Matplotlib to visualize the production and consumption of NADPH, NADH and ATP. % Maybe we won't include ATP bc it was the same  
Cofactor balances were visualized for simulations with both, biomass maximisation and NGAM minimisation, 
as the objective function. Information about production and consumption of specific cofactors, was extracted from solution fluxes using the COBRApy functions 
\vspace{-0.4cm} % \setlist{nolistsep}
\begin{itemize}[noitemsep]
    \item \verb|model.metabolites.metabolite.summary().producing_flux| and
    \item \verb|model.metabolites.metabolite.summary().consuming_flux|. 
\end{itemize}
\vspace{-0.4cm} % \setlist{nolistsep}
Reactions that had the same absolute value in producing and consuming fluxes were excluded. Results were plotted on a pie chart. The reactions that had a lower proportion than 5\% of the total flux of the corresponding cofactor, were summed together and represented in the sector `Other consuming' or `Other producing', respectively.

% Finally, for some models the central carbon metabolism fluxes were visualized with Escher \cite{King2015}. On Escher platform it is possible
% to create metabolic maps based on the reactions of a given GEM, which can then be fed with the simulated
% flux distributions to have a better overview of used metabolic pathways.
